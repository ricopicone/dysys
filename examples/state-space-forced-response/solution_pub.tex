\phantomsection\label{72b9b0a6}
Load Python packages

\phantomsection\label{457e325e}
\nointerlineskip\nointerlineskip\begin{minted}[autogobble,samepage]{python}
import numpy as np
import sympy as sp
import matplotlib.pyplot as plt
import dysys
from pprint import pprint
\end{minted}

\phantomsection\label{31f2d103}

\phantomsection\label{c52bb6f0}
\subsubsection{Define the System}\label{define-the-system}

Define a symbolic state-space matrix using the \mintinline{text}|dysys|
package as follows:

\phantomsection\label{31f1c682}
\nointerlineskip\nointerlineskip\begin{minted}[autogobble,samepage]{python}
A = [[-4, -3, 0], [0, -8, 4], [0, 0, -1]]
B = [[0], [1], [0]]
C = [[0, 1, 0]]
D = [[0]]
sys = dysys.sss(A, B, C, D)  # Create a symbolic state-space model
\end{minted}

\phantomsection\label{9ec9f9a9}
\subsubsection{Eigenvalues and
Stability}\label{eigenvalues-and-stability}

The eigenvalue matrix \mintinline{text}|L| can be found via the eig()
method:

\phantomsection\label{ed6af269}
\nointerlineskip\nointerlineskip\begin{minted}[autogobble,samepage]{python}
L, M = sys.eig()
print(f"Eigenvalues: {L.diagonal().tolist()}")
\end{minted}

\nointerlineskip\nointerlineskip\begin{minted}[autogobble,samepage]{text}
Eigenvalues: [[-8, -4, -1]]
\end{minted}

\phantomsection\label{f9608925}
The real parts of the eigenvalues are all negative; therefore, the
system is asymptotically stable.

\phantomsection\label{bf655b6a}
\subsubsection{Eigenvectors and the Modal
Matrix}\label{eigenvectors-and-the-modal-matrix}

The eigenvectors are stored in \mintinline{text}|M|:

\phantomsection\label{cabd61bc}
\nointerlineskip\nointerlineskip\begin{minted}[autogobble,samepage]{python}
print(M)
\end{minted}

\nointerlineskip\nointerlineskip\begin{minted}[autogobble,samepage]{text}
Matrix([[3/4, 1, -4/7], [1, 0, 4/7], [0, 0, 1]])
\end{minted}

\phantomsection\label{1a782013}
\subsubsection{State Transition Matrix}\label{state-transition-matrix}

The state transition matrix $\Phi(t)$ can be found as follows:

\phantomsection\label{a4b9b93a}
\nointerlineskip\nointerlineskip\begin{minted}[autogobble,samepage]{python}
t = sp.Symbol("t", nonnegative=True)  ## Solution valid for t >= 0
Phi = sys.state_transition_matrix(t)
pprint(Phi)
\end{minted}

\nointerlineskip\nointerlineskip\begin{minted}[autogobble,samepage]{text}
Matrix([
[exp(-4*t), -3*exp(-4*t)/4 + 3*exp(-8*t)/4, -4*exp(-t)/7 + exp(-4*t) - 3*exp(-8*t)/7],
[        0,                      exp(-8*t),              4*exp(-t)/7 - 4*exp(-8*t)/7],
[        0,                              0,                                  exp(-t)]])
\end{minted}

\phantomsection\label{fb0f8d0b}
\subsubsection{Forced State Response}\label{forced-state-response}

The forced state response for $t \ge 0$ can be found as follows:

\phantomsection\label{2adb3ba2}
\nointerlineskip\nointerlineskip\begin{minted}[autogobble,samepage]{python}
u_s = sp.Heaviside(t)
x_fo = sys.state_forced_response(t, u=u_s)
pprint(x_fo)
\end{minted}

\nointerlineskip\nointerlineskip\begin{minted}[autogobble,samepage]{text}
Matrix([
[-3/32 + 3*exp(-4*t)/16 - 3*exp(-8*t)/32],
[                      1/8 - exp(-8*t)/8],
[                                      0]])
\end{minted}

\phantomsection\label{532e6346}
\subsubsection{Forced Output Response}\label{forced-output-response}

\phantomsection\label{971f95db}
\nointerlineskip\nointerlineskip\begin{minted}[autogobble,samepage]{python}
y_fo = sys.output_forced_response(t, u=u_s)
pprint(y_fo)
\end{minted}

\nointerlineskip\nointerlineskip\begin{minted}[autogobble,samepage]{text}
Matrix([[1/8 - exp(-8*t)/8]])
\end{minted}

\phantomsection\label{f09a6196}
\subsubsection{Plot the Forced Output
Response}\label{plot-the-forced-output-response}

First, convert the symbolic expression to a NumPy function:

\phantomsection\label{c02b557d}
\nointerlineskip\nointerlineskip\begin{minted}[autogobble,samepage]{python}
y_fo_fun = sp.lambdify(t, y_fo, "numpy")
\end{minted}

\phantomsection\label{9d64b860}
Now create arrays to plot:

\phantomsection\label{be886d32}
\nointerlineskip\nointerlineskip\begin{minted}[autogobble,samepage]{python}
t_ = np.linspace(0, 5, 101)
y_fo_ = y_fo_fun(t_).flatten()
\end{minted}

\phantomsection\label{f269af9f}
Finally, plot:

\phantomsection\label{cfaff726}
\nointerlineskip\nointerlineskip\begin{minted}[autogobble,samepage]{python}
fig, ax = plt.subplots()
ax.plot(t_, y_fo_)
ax.set_xlabel("Time (s)")
ax.set_ylabel("Forced Output Response $y_\\text{fo}$")
plt.show()
\end{minted}

\phantomsection\label{6229d772}
\gdef\graphicslist{}%
\begin{figure}[htbp]
\centering
\includegraphics[]{examples/state-space-forced-response/figure-0.pgf}
\caption{}
\label{fig:state-space-forced-response-figure-0}
\end{figure}
